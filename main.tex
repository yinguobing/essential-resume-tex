\documentclass[10pt]{resume}

\usepackage{enumitem}
\setlist{topsep=0pt, parsep=0.35ex, leftmargin=1.5em} 

\begin{document}

% 个人信息
\begin{center}
    \name{图恒宇}
    \phone{+86 186-8686-6868} ~ \mail{hengyu@tu.com} ~ \blog{hengyu.tu} ~ \github{ghost} \\ [10pt]
\end{center}

% 概况
投身于移山计划,为人类文明的延续贡献自己的力量。

% 工作经历
\section{工作经历}
\textbf{量子计算研发制造中心 - 技术负责人 \hfill 地球地下城3号,2044/5 - 2058/1}\\
统筹量子计算与类人智能算法的耦合与实现。

\textbf{逐月基地MT01制造中心 - 总装工程师 \hfill 月球暗面区1号,2035/5 - 2044/1}
\begin{itemize}
    \item 负责逐月系列发动机之间的超低时延通讯技术开发。
    \item 流浪地球计划行星发动机人才储备工程负责人。
\end{itemize}

\textbf{数字生命研究所 - 数字生命架构师 \hfill 北京,2028/4 - 2035/4}\\
负责人类思想数字化的理论,以及神经细胞与硅原子的生物兼容性研究。


% 项目经历
\section{部分项目经历}
\textbf{移山计划}\\
量子计算实体550W,具备超大通量、实时并行与真随机特质的亚原子尺度规律推演单元。
\begin{itemize}
    \item 类人智能兼容性研发。
    \item 能源效率内置化,仅需要单个NUS聚变单元驱动。
\end{itemize}

\textbf{逐月计划}
\begin{itemize}
    \item 负责逐月一号卫星发动机的氦聚变核心模块开发。
    \item 月球网联发动机驱动状态的纳秒级同步技术开发。
\end{itemize}

\textbf{数字生命计划}
\begin{itemize}
    \item 脑机接口开发。
    \item 全封闭式低功耗自循环人体纳管仓开发。
\end{itemize}

% 教育经历
\section{教育经历}
\textbf{清华大学,博士研究生 \hfill 2023 - 2028} \\
生物与物理学院。研究方向为智慧生物神经元量子状态的可观测性与可操作性。

\textbf{清华大学,理学学士 \hfill 2019 - 2023} \\
人工智能专业。

% 技能与成就
\section{技能与成就}
\begin{itemize}
    \item 小型月地往返自驱动载具驾驶证,A级。
    \item 便携式聚变能源块操作许可证。
    \item UEG银河系超深液态地质环境探索许可证。
\end{itemize}

\end{document}
